\documentclass{article}
\usepackage{graphicx} % Required for inserting images
\usepackage{hyperref}


\title{Summer Linear Algebra Overview}

\begin{document}

\maketitle

\section{Quick Links}
\begin{itemize}
  \item \href{https://us06web.zoom.us/j/86241673980}{\textbf{Zoom room}}
  \item \href{https://www.overleaf.com/8251467867kkqxyvzcmndp#f019d2}{\textbf{Weekly problem presentations}}
  \item \href{https://www.overleaf.com/7498622626cjfdwmgbbcjq#7b94e3}{\textbf{Running Errata}}
\end{itemize}

\section{Overview}
This reading group will be a second pass at linear algebra, similar to how a first course in analysis is a second pass at calculus. We will rigorously work through the following topics:
\begin{itemize}
    \item linear spaces and dual spaces;
    \item linear mappings and matrices;
    \item determinant and trace;
    \item spectral theory;
    \item inner product space theory;
    \item calculus of matrix-valued functions.
\end{itemize}

The reading group will begin on May 19th and end on August 31st (15 weeks). Since we have the whole summer, we will go at a relatively leisurely pace: maybe five to ten hours a week of work. The idea is that you would be able to do this even with a full time job.

\section{Prerequisites}
Participants should be comfortable with basic linear algebra concepts, as one might reasonably have retained from a first computational pass at the subject. Some experience with mathematical proof is probably necessary.

\section{Textbooks}
All the books mentioned in this section can be found \href{https://drive.google.com/drive/folders/1chb49hv65hB8ZXcne0Ljfh3viO_SdOmW?usp=drive_link}{here}.

\subsection{Primary textbook}
Let's use as our primary textbook: \textbf{\emph{Linear Algebra and Its Applications}} by Peter Lax (RIP). I should highlight that the book is very fast-paced and takes a more analytic approach. If people want, we can also switch to a different book at some point.

\subsection{Supplementary textbooks}
There are many other good textbooks at this level. Below are the ones I know of:
\begin{itemize}
    \item \textit{Linear Algebra} by Hoffman and Kunze;
    \item \textit{Linear Algebra} by Serge Lang;
    \item \textit{Linear Algebra Done Right} by Sheldon Axler;
    \item \textit{Linear Algebra Done Wrong} by Sergei Treil;
    \item \textit{Finite-Dimensional Vector Space} by Paul Halmos;
    \item \textit{Linear Algebra} by Georgi Shilov;
    \item \textit{Linear Algebra} by Stephen Friedberg et al.
\end{itemize}

\section{Format}
Here's what we will do each week before the meeting:
\begin{enumerate}
    \item Work through around ten pages of Lax, as suggested in the reading schedule.
    \item Select one (or two, if it happens) problem that you think is interesting or difficult. This could be from Lax or from one of the textbooks above. (See also \href{https://www.dpmms.cam.ac.uk/study/IB/LinearAlgebra/}{these example sheets} and \href{https://ocw.mit.edu/courses/18-700-linear-algebra-fall-2013/pages/assignments/}{these problem sets}.) Write down the problem in \href{https://www.overleaf.com/8251467867kkqxyvzcmndp#f019d2}{\textbf{this Overleaf file}}, and prepare to walk through your solution if you have one.
    \item If you have time, try to solve other problems in the Overleaf file.
    \item If it's your turn, prepare to present the proof of two or three important theorems from this week.
\end{enumerate}

We will meet Sunday 3pm over zoom, for around 90 minutes. Here's how these may go:
\begin{itemize}
    \item One person (rotating weekly) summarizes definitions and proofs from this week.
    \item General discussion and Q\&A.
    \item Each participant presents one or two problems for the group.
\end{itemize}

\section{Reading Schedule}
Below is a provisional reading plan. Chapter and page number refer to Lax.

\begin{itemize}
  \item \textbf{May 25}: Linear space \& quotient space — Ch.~1, 1–12
  \item \textbf{June 1}: Dual space — Ch.~2, 13–18
  \item \textbf{June 8}: Linear mapping — Ch.~3, 19–31
  \item \textbf{June 15}: Matrices — Ch.~4, 32–43
  \item \textbf{June 22}: Determinant and trace — Ch.~5, 44–57
  \item \textbf{June 29}: Spectral theory I — Ch.~6, 58–67
  \item \textbf{July 6}: Spectral theory II — Ch.~6, 67–76
  \item \textbf{July 13}: Euclidean structure I — Ch.~7, 77–89
  \item \textbf{July 20}: Euclidean structure II — Ch.~7, 89–100
  \item \textbf{July 27}: Self‑adjoint mappings I — Ch.~8, 101–111
  \item \textbf{Aug 3}: Self‑adjoint mappings II — Ch.~8, 111–120
  \item \textbf{Aug 10}: Calculus I — Ch.~9, 121–129
  \item \textbf{Aug 17}: Calculus II — Ch.~9, 129–142
  \item \textbf{Aug 24}: Inequalities I — Ch.~10, 143–157
  \item \textbf{Aug 31}: Inequalities II — Ch.~10, 157–171
\end{itemize}

\end{document}

